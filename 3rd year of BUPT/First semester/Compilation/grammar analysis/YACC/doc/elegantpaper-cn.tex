%!TEX program = xelatex
% 完整编译: xelatex -> biber/bibtex -> xelatex -> xelatex
\documentclass[lang=cn,11pt,a4paper]{elegantpaper}

\title{语法分析程序的设计与实现YACC}
\author{杨晨 \\学号2021212171}
\institute{北京邮电大学 计算机学院}

% \version{0.10}
\date{\zhtoday}

% 本文档命令
\usepackage{array}
\usepackage{xcolor}
\newcommand{\ccr}[1]{\makecell{{\color{#1}\rule{1cm}{1cm}}}}

\usepackage{fontspec}

% 设置代码样式
% 定义关键字和注释的颜色
\definecolor{codegreen}{RGB}{34,139,34}
\definecolor{codegray}{RGB}{128,128,128}
\definecolor{codepurple}{RGB}{138,43,226}
\definecolor{backcolour}{RGB}{240,240,240}

% 定义代码样式
\lstset{
    language=C++, % 设置代码语言为C++
    basicstyle=\ttfamily, % 设置基本字体样式为等宽字体
    backgroundcolor=\color{gray!10}, % 设置代码块背景颜色为白色
    commentstyle=\color{green!60!black}, % 设置注释颜色
    keywordstyle=\color{blue}, % 设置关键字颜色
    % stringstyle=\color{orange}, % 设置字符串颜色
    showstringspaces=false, % 不显示字符串中的空格符
    breaklines=true, % 自动断行
    % numbers=left, % 行号显示在左侧
    numberstyle=\small\color{gray}, % 设置行号样式
    frame=single, % 绘制代码框
    rulecolor=\color{black}, % 设置代码框颜色
    captionpos=b, % 设置标题位置为底部
    tabsize=4, % 设置制表符宽度
    keywordstyle=[1]\color{blue}, % 设置关键字样式
    keywordstyle=[2]\color{purple}, % 设置扩展关键字样式
    keywordstyle=[3]\color{teal}, % 设置内置类型样式
    keywordstyle=[4]\color{magenta}, % 设置注解样式
    keywordstyle=[5]\color{orange}, % 设置预处理指令样式
    keywordstyle=[6]\color{cyan!60!black}, % 设置其他关键字样式
    keywordstyle=[7]\color{violet}, % 设置特殊关键字样式
}

% 纯文本样式
\lstdefinelanguage{text}{
    basicstyle=\fontsize{7pt}{9pt}\ttfamily,
    keywordstyle=\fontsize{7pt}{9pt}\ttfamily,
    commentstyle=\fontsize{7pt}{9pt}\ttfamily,
    stringstyle=\fontsize{7pt}{9pt}\ttfamily,
    escapeinside={(*@}{@*)},
    showstringspaces=false,
    breaklines=true,
    breakatwhitespace=true,
    tabsize=4
}

\begin{document}


\maketitle
% \tableofcontents

\section{概述}

\subsection{实验内容}

利用 YACC自动生成语法分析程序,实现对算术表达式的语法分析。要求所分析算数表达式由如下的文法产生

\label{grammar}
$$
\begin{aligned}
& E\rightarrow E+T | E-T | T \\
& T\rightarrow T*F | T/F | F \\
& F\rightarrow (E) | num
\end{aligned}
$$

要求在对输入的算术表达式进行分析的过程中,依次输出所采用的产生式。

实验要求和实现方法要求如下:

\begin{itemize}
    \item 根据给定文法,编写 YACC 说明文件,调用 LEX 生成的词法分析程序。
\end{itemize}

\subsection{开发环境}

\begin{itemize}
    \item Windows10
    \item Visual Studio Code
    \item Win flex-bison
\end{itemize}

\section{LEX程序的功能模块划分}

\subsection{常见的标记模式和正则表达式}


\begin{lstlisting}[language=text]
O    [0-7]  
D    [0-9] 
NZ   [1-9] 
L    [a-zA-Z_] 
A    [a-zA-Z_0-9] 
H    [a-fA-F0-9] 
HP   (0[xX]) 
E    ([Ee][+-]?{D}+) 
P    ([Pp][+-]?{D}+) 
FS   (f|F|l|L) 
IS   (((u|U)(l|L|ll|LL)?)|((l|L|ll|LL)(u|U)?)) 
CP   (u|U|L) 
SP   (u8|u|U|L) 
ES   (\\(['"\?\\abfnrtv]|[0-7]{1,3}|x[a-fA-F0-9]+)) 
NE   [^"\\\n] 
WS   [ \t\v\f] 
end  [$\r]

\end{lstlisting}

这段代码用于定义一些常见的标记模式和正则表达式。下面是对每个定义的模式的详细解释:

\begin{itemize}
    \item O: 匹配八进制数字的字符集,范围从0到7。
    \item D: 匹配十进制数字的字符集,范围从0到9。
    \item NZ: 匹配非零的十进制数字的字符集,范围从1到9。
    \item L: 匹配字母、下划线的字符集,包括小写字母、大写字母和下划线。
    \item A: 匹配字母、数字、下划线的字符集,包括小写字母、大写字母、数字和下划线。
    \item H: 匹配十六进制数字的字符集,包括小写字母、大写字母和数字。
    \item HP: 匹配十六进制数的前缀,即以"0x"或"0X"开头。
    \item E: 匹配浮点数中的指数部分,包括可选的正负号,后面跟着十进制数字。
    \item P: 匹配浮点数中的十六进制指数部分,包括可选的正负号,后面跟着十六进制数字。
    \item FS: 匹配浮点数类型后缀,可以是"f"、"F"、"l"或"L"。
    \item IS: 匹配整数类型后缀,可以是各种组合,如"u"、"U"、"l"、"L"、"ll"、"LL"等。
    \item CP: 匹配字符类型后缀,可以是"u"、"U"或"L"。
    \item SP: 匹配字符串类型前缀,可以是"u8"、"u"、"U"或"L"。
    \item ES: 匹配转义序列,包括转义字符(如"\char`\\n"、"\char`\\t"等)、八进制字符(如"\char`\\377")和十六进制字符(如"\char`\\xFF")。
    \item NE: 匹配非转义字符和换行符的字符集。
    \item WS: 匹配空白字符,包括空格、制表符、垂直制表符和换页符。
    \item END: 匹配'\$'符号,代表输入结束
\end{itemize}

\subsection{C语言部分}
\begin{lstlisting}[language=C]
%{

#include <stdio.h>
#include "parser.tab.h"

%}
\end{lstlisting}


这段代码是使用Lex工具编写的C语言词法分析器。让我们对其进行详细解释:

%{ 和 %}:这些符号用于将C代码插入到Lex规则之间,以便在生成的词法分析器中包含所需的C代码。

\lstinline{#include <stdio.h>}这是C语言中的预处理指令,用于包含标准输入输出库的头文件\lstinline{#include <stdio.h>},以便在词法分析器中使用\lstinline{printf}和其他相关函数。

\lstinline{#include "parser.tab.h}是我们用YACC编译 .h 文件后,产生的.h文件,在后续的链接编译中有重要作用


\subsection{翻译规则部分}

\subsubsection{继续解析}
\begin{lstlisting}
"//".* { return -1; }  /* single-line comment */
"#".+  { return -1; }  /* preprocessor directive */
{WS}   {return -1;}    /* white space */
\end{lstlisting}

在Lex中,规则可以返回一个整数值,用于指示词法分析器采取的下一个操作。通常情况下,返回-1表示继续解析,而返回其他值可能表示要进行一些特定的操作或者终止解析过程。上述代码的作用如下:
\begin{itemize}
    \item 当遇到以\lstinline{"//"}开头的注释时要执行的操作。它匹配从\lstinline{"//"}开始的任意字符序列(\lstinline{.*}表示匹配任意字符零次或多次)
    \item 当遇到以\lstinline{"#"}开头的预处理指令时要执行的操作。它匹配以\lstinline{"#"}开头的任意字符序列(\lstinline{.+}表示匹配任意字符至少一次)
    \item 当遇到空字符时,跳过。
\end{itemize}

\subsubsection{整数常量}
\begin{lstlisting}
{HP}{H}+{IS}?			    { return I_CONSTANT; }  /* constants */
{NZ}{D}*{IS}?			    { return I_CONSTANT; }
"0"{O}*{IS}?		    	{ return I_CONSTANT; }
\end{lstlisting}

这段Lex代码用于匹配并返回C语言中的整数常量(integer constants)。它包含了三个规则:
\begin{itemize}
    \item 匹配十六进制整数常量的模式。它以HP(0x或0X)作为起始,后面跟随一个或多个十六进制数字(H),并可选择性地包含一个后缀(IS)来指示整数的类型。
    \item 匹配十六进制整数常量的模式。它以HP(0x或0X)作为起始,后面跟随一个或多个十六进制数字(H),并可选择性地包含一个后缀(IS)来指示整数的类型。
    \item 匹配八进制整数常量的模式。它以字符0作为起始,后面可以跟随零个或多个八进制数字(O),并可选择性地包含一个后缀(IS)来指示整数的类型。
\end{itemize}

\subsubsection{浮点常量}

\begin{lstlisting}
{D}+{E}{FS}?		    		{ return F_CONSTANT; }
{D}*"."{D}+{E}?{FS}?			{ return F_CONSTANT; }
{D}+"."{E}?{FS}?		    	{ return F_CONSTANT; }
{HP}{H}+{P}{FS}?		    	{ return F_CONSTANT; }
{HP}{H}*"."{H}+{P}{FS}?			{ return F_CONSTANT; }
{HP}{H}+"."{P}{FS}?		    	{ return F_CONSTANT; }
\end{lstlisting}

这段Lex代码用于匹配并返回C语言中的浮点数常量(floating-point constants)。它包含了六个规则,如下:

\begin{itemize}
    \item 以一个或多个数字(D)开头,接着是一个指数部分,由字母E表示,后面跟着一个可选的符号和一个或多个数字。最后,可能会包含一个可选的浮点数后缀(FS)。
    \item 以零个或多个数字(D)开头,接着是一个小数点,然后是一个或多个数字。可选地,可以包含一个指数部分,由字母E表示,后面跟着一个可选的符号和一个或多个数字。最后,可能会包含一个可选的浮点数后缀(FS)
    \item 以一个或多个数字(D)开头,接着是一个小数点。可选地,可以包含一个指数部分,由字母E表示,后面跟着一个可选的符号和一个或多个数字。最后,可能会包含一个可选的浮点数后缀(FS)。
    \item 以十六进制前缀(HP)开头,后面跟着一个或多个十六进制数字(H),然后是一个指数部分,由字母P表示。最后,可能会包含一个可选的浮点数后缀(FS)。
    \item 以十六进制前缀(HP)开头,后面跟着零个或多个十六进制数字(H),然后是一个小数点,接着是一个或多个十六进制数字。最后,需要包含一个指数部分,由字母P表示。最后,可能会包含一个可选的浮点数后缀(FS)。
    \item 以十六进制前缀(HP)开头,后面跟着一个或多个十六进制数字(H),然后是一个小数点,接着是一个指数部分,由字母P表示。最后,可能会包含一个可选的浮点数后缀(FS)。
\end{itemize}

这些规则按照特定的模式匹配浮点数常量,并在匹配成功时返回F\_CONSTANT标识符,以便在Lex文件的其他规则中对浮点数常量进行特殊处理或进一步分析。

\subsubsection{标点符号}

\begin{lstlisting}
{end}                           { return END; }
"("                             { return LP; }
")"                             { return RP; }
"+"                             { return PLUS; }
"-"                             { return MINUS; }
"*"                             { return TIMES; }
"/"                             { return DIV; }
\end{lstlisting}

这段Lex代码片段用于匹配和识别我们需要的基本运算符号。

\begin{itemize}
    \item 如果是"(",则标记为LP
    \item 如果是")",则标记为RP
    \item 如果是"+",则标记为PLUS
    \item 如果是"-",则标记为MINUS
    \item 如果是"*",则标记为LP
    \item 如果是"/",则标记为LP
    \item 如果是"\$",则标记为END
\end{itemize}

总之,这段代码定义了一系列模式,用于匹配和识别各种标点符号,根据不同的标点,返回不同的标记模式

\subsubsection{错误}

\begin{lstlisting}
/* errors */
{D}+({A}|\.)+    { yyerror("illegal name"); return -1; } 
{SP}?\"({NE}|{ES})*{WS}*  { yyerror("unclosed string"); return -1; }
.	    		 { yyerror("unexpected character"); return -1; }
\end{lstlisting}

这部分定义了三个规则:
\begin{itemize}
    \item {D}+({A}|\.)+:匹配非法的名称。它匹配一个或多个数字({D}+),后跟一个字母或点号({A}|.)+)。当匹配到这种模式时,词法分析器将输出一个错误消息“illegal name”。
    \item {SP}?\"({NE}|{ES})*{WS}*:匹配未关闭的字符串常量。它匹配一个可选的前缀({SP}?),后跟一个双引号,然后匹配零个或多个非转义字符(NE)或转义字符(ES),再匹配零个或多个空格字符,由于缺少结束双引号",词法分析器将输出一个错误信息"unclosed string"
    \item 匹配除换行符外的任意字符。当词法分析器遇到任何无法匹配其他规则的字符时,它会匹配这个规则。当匹配到这种模式时,词法分析器将返回一个错误消息"unexpected character",表示遇到了意外的字符。
\end{itemize}


\section{YACC程序的功能模块划分}

\subsection{C语言部分}
\begin{lstlisting}
%{
#include <stdio.h>
# define ACC 1
extern int yylex();
int yyerror(const char* s);
int success = 1;
%}
\end{lstlisting}

这部分代码是 Yacc程序的声明和全局定义部分。以下是对每个部分的解释:
\begin{itemize}
    \item \lstinline"%{ ... %}": 这是 Yacc 源文件的可选部分,它允许在其中插入 C 代码。在这个部分中,可以包含任意的 C 头文件、宏定义和全局变量的声明。
    \item \lstinline{#include <stdio.h>}: 这是一个包含标准输入输出函数的头文件。它使得我们可以使用 \lstinline{printf} 和其他标准输入输出函数。
    \item \lstinline{#define ACC 1}: 这里定义了一个名为 \lstinline{ACC} 的宏常量,它的值为 \lstinline{1}。这个宏常量可能在后面的代码中用于表示解析成功的状态。
    \item \lstinline{extern int yylex();}: 这是一个函数声明,用于声明一个名为 \lstinline{yylex} 的外部函数。这个函数通常是由词法分析器生成的函数,用于获取输入并返回词法单元的类型。
    \item \lstinline{int yyerror(const char* s);}: 这是一个函数声明,用于声明一个名为 \lstinline{yyerror} 的函数。这个函数通常是由语法分析器调用的错误处理函数,用于处理语法错误并输出错误消息。
    \item \lstinline{int success = 1;}: 这是一个全局变量的定义,名为 \lstinline{success},并初始化为 \lstinline{1}。这个变量可能在程序的其他部分用于表示解析的成功与否的状态。
\end{itemize}


总之,这部分代码主要用于包含必要的头文件、声明外部函数和全局变量,以及为程序提供一些常量和错误处理函数。这些声明和定义为后续的 Yacc 规则和主程序提供了必要的基础。

\subsection{语法规则}

\begin{lstlisting}
%token PLUS MINUS TIMES DIV LP RP I_CONSTANT F_CONSTANT END
%start S

%%
S : E END { printf("S -> E\n"); return ACC; }
  ;

E: E PLUS T { printf("E -> E + T\n"); }
 | E MINUS T { printf("E -> E - T\n"); }
 | T { printf("E -> T\n"); }
 ;

T: T TIMES F { printf("T -> T * F\n"); }
 | T DIV F { printf("T -> T / F\n"); }
 | F { printf("T -> F\n"); }
 ;

F: LP E RP { printf("F -> (E)\n"); }
 | I_CONSTANT { printf("F -> num (I_CONSTANT)\n"); }
 | F_CONSTANT { printf("F -> num (F_CONSTANT)\n"); }
 ;

%%
\end{lstlisting}

这部分代码是 Yacc 程序的语法规则部分。它定义了语法规则和对应的动作,用于构建语法树或执行相关操作。下面是对每个部分的解释:

\begin{itemize}
    \item \%token PLUS MINUS TIMES DIV LP RP I\_CONSTANT F\_CONSTANT END
    
    这一行指定了词法单元的符号名称,也可以称为终结符号。每个符号名称代表了一个特定的词法单元类型。在这个例子中,定义了 PLUS、MINUS、TIMES、DIV、LP、RP、I\_CONSTANT、F\_CONSTANT 和 END 这些终结符号。
    
    \item \%start S
    
    这一行指定了起始符号,也就是语法分析的起点。在这个例子中,起始符号为 S。
    \item S : E END { \lstinline{printf("S -> E\n"); return ACC; }}
    
    这是一个语法规则,定义了 S 的产生式。它表示 S 可以推导出 E \$。在推导成功时,执行动作 { \lstinline{printf("S -> E\n"); return ACC; } },这里的动作是输出一条消息,并返回 ACC 状态。
    
    \item E: E PLUS T { \lstinline{printf("E -> E + T\n"); }}
    
    这是一个语法规则,定义了 E 的产生式。它表示 E 可以推导出 E + T。在推导成功时,执行动作 { \lstinline{printf("E -> E + T\n"); }},这里的动作是输出一条消息。
    
    \item | E MINUS T { \lstinline{printf("E -> E - T\n");} }
    
    这是 E 的另一个产生式,表示 E 可以推导出 E - T。在推导成功时,执行动作 { \lstinline{printf("E -> E - T\n");} },这里的动作是输出一条消息。
    
    \item | T { \lstinline{printf("E -> T\n");} }
    
    这是 E 的第三个产生式,表示 E 可以推导出 T。在推导成功时,执行动作 { \lstinline{printf("E -> T\n");} },这里的动作是输出一条消息。
    
    \item T: T TIMES F { \lstinline{printf("T -> T * F\n");} }
    
    这是 T 的一个产生式,表示 T 可以推导出 T * F。在推导成功时,执行动作 { \lstinline{printf("T -> T * F\n");} },这里的动作是输出一条消息。
    
    \item | T DIV F { \lstinline{printf("T -> T / F\n");} }
    
    这是 T 的另一个产生式,表示 T 可以推导出 T / F。在推导成功时,执行动作 {\lstinline{printf("T -> T / F\n");}  },这里的动作是输出一条消息。
    
    \item | F { \lstinline{printf("T -> F\n");} }
    
    这是 T 的第三个产生式,表示 T 可以推导出 F。在推导成功时,执行动作 { \lstinline{printf("T -> F\n");} },这里的动作是输出一条消息。

    \item F: LP E RP { \lstinline{printf("F -> (E)\n");} }
    
    这是 F 的一个产生式,表示 F 可以推导出 ( E )。在推导成功时,执行动作 { \lstinline{printf("F -> (E)\n");} },这里的动作是输出一条消息。
    
    \item | I\_CONSTANT { \lstinline{printf("F -> num (I_CONSTANT)\n");} }
    
    这是 F 的另一个产生式,表示 F 可以推导出 整数num 。在推导成功时,执行动作 { \lstinline{printf("F -> num (I_CONSTANT)\n");} },这里的动作是输出一条消息。

    \item | F\_CONSTANT { \lstinline{printf("F -> num (F_CONSTANT)\n");} }
    
    这是 F 的第三个产生式,表示 F 可以推导出 浮点数num。在推导成功时,执行动作 { \lstinline{printf("F -> num (F_CONSTANT)\n");} },这里的动作是输出一条消息。
    
    \item \%\%: 这是规则部分的结束符号,用于标识语法规则的结束。
\end{itemize}


\subsection{主函数与错误处理}

\begin{lstlisting}
int main(){
    yyparse();
    if (success == 1)
        printf("\033[32mParsing done.\033[0m\n");
    return 0;
}

int yyerror(const char *msg)
{
	extern int yylineno;
	printf("\033[31mParsing Failed\nLine Number: %d %s\033[0m\n", yylineno, msg);
    success = 0;
	return 0;
}
\end{lstlisting}

这部分代码包含了主函数 \lstinline{main} 和错误处理函数 \lstinline{yyerror}。
\begin{itemize}
    \item \lstinline{int main()}: 这是程序的主函数。它调用了 \lstinline{yyparse()} 函数,该函数是由 Yacc 自动生成的用于执行语法分析的函数。\lstinline{yyparse()} 函数将根据之前定义的语法规则对输入进行语法分析,并根据语法规则中的动作执行相应的操作。在这个例子中,\lstinline{yyparse()} 函数完成语法分析后,通过检查 \lstinline{success} 变量的值来判断解析是否成功。如果 \lstinline{success} 的值为 1,表示解析成功,将输出一条成功的消息;否则,表示解析失败。

    \item \lstinline{int yyerror(const char *msg)}: 这是错误处理函数 \lstinline{yyerror} 的实现。当语法分析过程中发生错误时,Yacc 会调用这个函数。函数的参数 \lstinline{msg} 是错误消息的字符串。在这个例子中,\lstinline{yyerror} 函数使用 \lstinline{printf} 函数输出错误消息,并包含当前发生错误的行号\lstinline{yylineno}。同时,它将 \lstinline{success} 变量的值设置为 0,表示解析失败。
\end{itemize}


总之,这部分代码用于调用语法分析函数 \lstinline{yyparse()},并根据解析结果输出相应的消息。同时,它还实现了错误处理函数 \lstinline{yyerror},用于在发生语法错误时输出错误消息并记录解析失败的状态。

\section{使用说明}

\subsection{Bison编译程序}
利用win\_bison可以编译写好的.y文件,生成对应的.h和.c代码
\begin{lstlisting}
>  D:\win_flex_bison-latest\win_bison.exe -d parser.y    
\end{lstlisting}

\subsection{LEX编译程序}
利用win\_flex编译写好的.l文件,生成对应的.c代码
\begin{lstlisting}
> D:\win_flex_bison-latest\win_flex.exe --wincompat lexcial.l
\end{lstlisting}

\subsection{C语言混合编译程序}
将生成的2个.c进行链接编译,生成可执行文件parser.exe
\begin{lstlisting}
> gcc lex.yy.c parser.tab.c -o parser
\end{lstlisting}


\section{测试}

\subsection{测试 1+2}
该测试集用于测试一个简单的算数表达式能否被正确识别

\subsubsection{输出结果}

\begin{lstlisting}[language=text]
> ./parser.exe
1+2$
F -> num (I_CONSTANT)        
T -> F
E -> T
F -> num (I_CONSTANT)        
T -> F
E -> E + T
S -> E
(*@\textcolor{green!60!black}{Parsing done.}@*)
\end{lstlisting}


\subsubsection{输出结果分析}

给定的语法分析程序的输出结果表明,输入字符串"1+2\$"符合给定的文法规则。根据给出的输出结果,可以得出以下简单的输出分析总结:
\begin{enumerate}
    \item 输入表达式为 1+2\$
    \item 首先,识别到 1,将其作为 I\_CONSTANT(整数常量)规约为 F。
    \item 接着,识别到 2,同样将其作为 I\_CONSTANT 规约为 F。
    \item 将 F 规约为 T。
    \item 再次将 F 规约为 T。
    \item 将 T 规约为 E。
    \item 最后,将 E + T 规约为 E。
    \item 整个表达式成功地被语法分析器归约为起始符号 S。
\end{enumerate}


因此,可以总结出以上输出的分析结果为:给定的输入表达式 1+2\$ 符合定义的语法规则,并且成功地被规约为起始符号 S。

\subsection{测试 1+2*(3-(4/0))}

\subsubsection{输出结果}

\begin{lstlisting}[language=text]
1+2*(3-(4/0))$
F -> num (I_CONSTANT)
T -> F
E -> T
F -> num (I_CONSTANT)
T -> F
F -> num (I_CONSTANT)
T -> F
E -> T
F -> num (I_CONSTANT)
T -> F
F -> num (I_CONSTANT)
T -> T / F
E -> T
F -> (E)
T -> F
E -> E - T
F -> (E)
T -> T * F
E -> E + T
S -> E
(*@\textcolor{green!60!black}{Parsing done.}@*)
\end{lstlisting}

\subsubsection{输出结果分析}

通过一系列操作执行,程序成功地完成了对输入字符串"1+2*(3-(4/0))"的语法分析。

根据给出的输出结果,可以得出以下简单的输出分析总结:
\begin{enumerate}
    \item 首先,识别到 1,将其作为 I\_CONSTANT(整数常量)规约为 F。
    \item 接着,识别到 2,同样将其作为 I\_CONSTANT 规约为 F。
    \item 将 F 规约为 T。
    \item 继续识别到 3,将其作为 I\_CONSTANT 规约为 F。
    \item 再次将 F 规约为 T。
    \item 接下来,识别到 4,将其作为 I\_CONSTANT 规约为 F。
    \item 将 F 规约为 T。
    \item 将 T / F 规约为 T。
    \item 再次将 F 规约为 T。
    \item 将 (E) 规约为 F。
    \item 将 F 规约为 T。
    \item 将 T * F 规约为 T。
    \item 将 T 规约为 E。
    \item 将 (E) 规约为 F。
    \item 将 F 规约为 T。
    \item 将 T - F 规约为 T。
    \item 将 T 规约为 E。
    \item 最后,将 E + T 规约为 E。
    \item 整个表达式成功地被语法分析器归约为起始符号 S。
\end{enumerate}

因此,可以总结出以上输出的分析结果为:给定的输入表达式 1+2*(3-(4/0))\$ 符合定义的语法规则,并且成功地被规约为起始符号 S。

\subsection{测试 1+2*/(3-4/0))}

\subsubsection{输出结果}

\begin{lstlisting}[language=text]
1+2*/(3-4/0))
F -> num (I_CONSTANT)
T -> F
E -> T
F -> num (I_CONSTANT)
T -> F
(*@\textcolor{red}{Parsing Failed}@*) 
(*@\textcolor{red}{Line Number: 1 syntax error}@*)
\end{lstlisting}

\subsubsection{输出结果分析}

根据给出的输出结果,可以得出以下简单的输出分析总结:
\begin{enumerate}
    \item 首先,识别到 1,将其作为 I\_CONSTANT(整数常量)规约为 F。
    \item 接着,识别到 2,同样将其作为 I\_CONSTANT 规约为 F。
    \item 将 F 规约为 T。
    \item 遇到 /,但是没有合适的终结符号可以规约,因此语法分析失败。
    \item 输出结果指示语法分析失败,给定的输入表达式存在语法错误。
    \item 在行号 1 处发现语法错误。
\end{enumerate}


因此,可以总结出以上输出的分析结果为:给定的输入表达式 `1+2*/(3-4/0))` 存在语法错误,无法被正确规约。语法错误发生在行号 1 处。


\section{实验总结}

在本次实验中,我使用了LEX工具和YACC工具编写了一个语法分析程序,以实现对C语言中整型和浮点型常量的词法分析。

首先,我仔细阅读了C11的ISO标准,特别关注了标准中对常量的描述。这帮助我明确了C语言中整型和浮点型常量的定义和规则。

接着,我根据标准中给出的语法和我的课堂学习,设计了用于匹配整型和浮点型常量的正则表达式。这些正则表达式可以用于LEX工具生成词法分析器。

在设计词法分析程序时,我预先定义了一系列常用的正则表达式,以便后续的词法分析可以直接使用这些工具。这些正则表达式包含了整型和浮点型常量的模式,例如匹配整数、十六进制数、指数表示法等。此外,还考虑了C语言中常量的后缀,如L、U、F等。

接下来,我根据C语言的语法规则,使用YACC工具编写了相应的语法规约程序。这样,我可以定义C语言中的语法规则,包括算术表达式的结构等。

为了提高程序的鲁棒性,我还编写了简单的错误处理函数,用于处理词法分析和语法分析过程中可能出现的错误情况。这些错误处理函数可以帮助我在程序出现错误时给出相应的提示信息。

通过这次实验,我深入理解了语法分析的流程和相关知识点。通过使用LEX和YACC工具,我能够更方便地实现词法分析和语法分析,并且对C语言中整型和浮点型常量的词法规则有了更牢固的掌握。我还能通过自定义的错误处理函数,增强程序的健壮性,提供更好的错误提示和恢复机制。


% \subsection{全局选项}
% 此模板定义了一个语言选项 \lstinline{lang},可以选择英文模式 \lstinline{lang=en}(默认)或者中文模式 \lstinline{lang=cn}。当选择中文模式时,图表的标题引导词以及参考文献,定理引导词等信息会变成中文。你可以通过下面两种方式来选择语言模式:
% \begin{lstlisting}
% \documentclass[lang=cn]{elegantpaper} % or
% \documentclass{cn}{elegantpaper} 
% \end{lstlisting}

% \textbf{注意:} 英文模式下,由于没有添加中文宏包,无法输入中文。如果需要输入中文,可以通过在导言区引入中文宏包 \lstinline{ctex} 或者加入 \lstinline{xeCJK} 宏包后自行设置字体。 
% \begin{lstlisting}
% \usepackage[UTF8,scheme=plain]{ctex}
% \end{lstlisting}

% \subsection{数学字体选项}

% 本模板定义了一个数学字体选项(\lstinline{math}),可选项有三个:
% \begin{enumerate}
%   \item \lstinline{math=cm}(默认),使用 \LaTeX{} 默认数学字体(推荐,无需声明);
%   \item \lstinline{math=newtx},使用 \lstinline{newtxmath} 设置数学字体(潜在问题比较多)。
%   \item \lstinline{math=mtpro2},使用 \lstinline{mtpro2} 宏包设置数学字体,要求用户已经成功安装此宏包。
% \end{enumerate}

% \subsection{中文字体选项}

% 模板提供中文字体选项 \lstinline{chinesefont},可选项有
% \begin{enumerate}
%   \item \lstinline{ctexfont}:默认选项,使用 \lstinline{ctex} 宏包根据系统自行选择字体,可能存在字体缺失的问题,更多内容参考 \lstinline{ctex} 宏包\href{https://ctan.org/pkg/ctex}{官方文档}\footnote{可以使用命令提示符,输入 \lstinline{texdoc ctex} 调出本地 \lstinline{ctex} 宏包文档}。
%   \item \lstinline{founder}:方正字体选项(\textbf{需要安装方正字体}),后台调用 \lstinline{ctex} 宏包并且使用 \lstinline{fontset=none} 选项,然后设置字体为方正四款免费字体,方正字体下载注意事项见后文,用户只需要安装方正字体即可使用该选项。
%   \item \lstinline{nofont}:后台会调用 \lstinline{ctex} 宏包并且使用 \lstinline{fontset=none} 选项,不设定中文字体,用户可以自行设置中文字体,具体见后文。
% \end{enumerate}

% \subsubsection{方正字体选项}
% 由于使用 \lstinline{ctex} 宏包默认调用系统已有的字体,部分系统字体缺失严重,因此,用户希望能够使用其它字体,我们推荐使用方正字体。方正的{\songti 方正书宋}、{\heiti 方正黑体}、{\kaishu 方正楷体}、{\fangsong 方正仿宋}四款字体均可免费试用,且可用于商业用途。用户可以自行从\href{http://www.foundertype.com/}{方正字体官网}下载此四款字体,在下载的时候请\textbf{务必}注意选择 GBK 字符集,也可以使用 \href{https://www.latexstudio.net/}{\LaTeX{} 工作室}提供的\href{https://pan.baidu.com/s/1BgbQM7LoinY7m8yeP25Y7Q}{方正字体,提取码为:njy9} 进行安装。安装时,{\kaishu Win 10 用户请右键选择为全部用户安装,否则会找不到字体。}

% \begin{figure}[!htb]
% \centering
% \includegraphics[width=0.9\textwidth]{founder.png}
% \end{figure}

% \subsubsection{其他中文字体}
% 如果你想完全自定义字体\footnote{这里仍然以方正字体为例。},你可以选择 \lstinline{chinesefont=nofont},然后在导言区设置即可,可以参考下方代码:
% \begin{lstlisting}
% \setCJKmainfont[BoldFont={FZHei-B01},ItalicFont={FZKai-Z03}]{FZShuSong-Z01}
% \setCJKsansfont[BoldFont={FZHei-B01}]{FZKai-Z03}
% \setCJKmonofont[BoldFont={FZHei-B01}]{FZFangSong-Z02}
% \setCJKfamilyfont{zhsong}{FZShuSong-Z01}
% \setCJKfamilyfont{zhhei}{FZHei-B01}
% \setCJKfamilyfont{zhkai}[BoldFont={FZHei-B01}]{FZKai-Z03}
% \setCJKfamilyfont{zhfs}[BoldFont={FZHei-B01}]{FZFangSong-Z02}
% \newcommand*{\songti}{\CJKfamily{zhsong}}
% \newcommand*{\heiti}{\CJKfamily{zhhei}}
% \newcommand*{\kaishu}{\CJKfamily{zhkai}}
% \newcommand*{\fangsong}{\CJKfamily{zhfs}}
% \end{lstlisting}



% \subsection{自定义命令}
% 此模板并没有修改任何默认的 \LaTeX{} 命令或者环境\footnote{目的是保证代码的可复用性,请用户关注内容,不要太在意格式,这才是本工作论文模板的意义。}。另外,我自定义了 4 个命令:
% \begin{enumerate}
%   \item \lstinline{\email}:创建邮箱地址的链接,比如 \email{ddswhu@outlook.com};
%   \item \lstinline{\figref}:用法和 \lstinline{\ref} 类似,但是会在插图的标题前添加 <\textbf{图 n}> ;
%   \item \lstinline{\tabref}:用法和 \lstinline{\ref} 类似,但是会在表格的标题前添加 <\textbf{表 n}>;
%   \item \lstinline{\keywords}:为摘要环境添加关键词。
% \end{enumerate}

% \subsection{参考文献}

% 文献部分,本模板调用了 biblatex 宏包,并提供了 biber(默认) 和 bibtex 两个后端选项,可以使用 \lstinline{bibend} 进行修改:

% \begin{lstlisting}
%   \documentclass[bibtex]{elegantpaper}
%   \documentclass[bibend=bibtex]{elegantpaper}
% \end{lstlisting}

% 关于文献条目(bib item),你可以在谷歌学术,Mendeley,Endnote 中取,然后把它们添加到 \lstinline{reference.bib} 中。在文中引用的时候,引用它们的键值(bib key)即可。

% 为了方便文献样式修改,模板引入了 \lstinline{bibstyle} 和 \lstinline{citestyle} 选项,默认均为数字格式(numeric),参考文献示例:\cite{cn1,en2,en3} 使用了中国一个大型的 P2P 平台(人人贷)的数据来检验男性投资者和女性投资者在投资表现上是否有显著差异。

% 如果需要设置为国标 GB7714-2015,需要使用:
% \begin{lstlisting}
%   \documentclass[citestyle=gb7714-2015, bibstyle=gb7714-2015]{elegantpaper} 
% \end{lstlisting}

% 如果需要添加排序方式,可以在导言区加入
% \begin{lstlisting}
%   \ExecuteBibliographyOptions{sorting=ynt}
% \end{lstlisting}

% 启用国标之后,可以加入 \lstinline{sorting=gb7714-2015}。


% \section{使用 newtx 系列字体}

% 如果需要使用原先版本的 \lstinline{newtx} 系列字体,可以通过显示声明数学字体:

% \begin{lstlisting}
% \documentclass[math=newtx]{elegantpaper}
% \end{lstlisting}

% \subsection{连字符}

% 如果使用 \lstinline{newtx} 系列字体宏包,需要注意下连字符的问题。
% \begin{equation}
%   \int_{R^q} f(x,y) dy.\emph{of\kern0pt f}
% \end{equation}

% \begin{lstlisting}
% \begin{equation}
%   \int_{R^q} f(x,y) dy.\emph{of \kern0pt f}
% \end{equation}
% \end{lstlisting}

% \subsection{宏包冲突}

% 有用户反馈模板在使用 \lstinline{yhmath} 以及 \lstinline{esvect} 等宏包时会报错:
% \begin{lstlisting}
% LaTeX Error:
%    Too many symbol fonts declared.
% \end{lstlisting}

% 原因是在使用 \lstinline{newtxmath} 宏包时,重新定义了数学字体用于大型操作符,达到了 {\heiti 最多 16 个数学字体} 的上限,在调用其他宏包的时候,无法新增数学字体。为了减少调用非常用宏包,在此给出如何调用 \lstinline{yhmath} 以及 \lstinline{esvect} 宏包的方法。

% 请在 \lstinline{elegantpaper.cls} 内搜索 \lstinline{yhmath} 或者 \lstinline{esvect},将你所需要的宏包加载语句\textit{取消注释}即可。


% \section{常见问题 FAQ}

% \begin{enumerate}[label=\arabic*).]
%   \item \textit{如何删除版本信息?}\\
%     导言区不写 \lstinline|\version{x.xx}| 即可。
%   \item \textit{如何删除日期?}\\
%     需要注意的是,与版本 \lstinline{\version} 不同的是,导言区不写或注释 \lstinline{\date} 的话,仍然会打印出当日日期,原因是 \lstinline{\date} 有默认参数。如果不需要日期的话,日期可以留空即可,也即 \lstinline|\date{}|。
%   \item \textit{如何获得中文日期?}\\
%     为了获得中文日期,必须在中文模式下\footnote{英文模式下,由于未加载中文宏包,无法输入中文。},使用 \lstinline|\date{\zhdate{2019/10/11}}|,如果需要当天的汉化日期,可以使用 \lstinline|\date{\zhtoday}|,这两个命令都来源于 \href{https://ctan.org/pkg/zhnumber}{\lstinline{zhnumber}} 宏包。
%   \item \textit{如何添加多个作者?}\\
%     在 \lstinline{\author} 里面使用 \lstinline{\and},作者单位可以用 \lstinline{\\} 换行。
%     \begin{lstlisting}
%     \author{author 1\\ org. 1 \and author 2 \\ org. 2 }
%     \end{lstlisting}
%   \item \textit{如何添加中英文摘要?}\\
%     请参考 \href{https://github.com/ElegantLaTeX/ElegantPaper/issues/5}{GitHub::ElegantPaper/issues/5}
% \end{enumerate}


% \section{致谢}

% 特别感谢 \href{https://github.com/sikouhjw}{sikouhjw} 和 \href{https://github.com/syvshc}{syvshc}  长期以来对于 Github 上 issue 的快速回应,以及各个社区论坛对于 ElegantLaTeX 相关问题的回复。特别感谢 ChinaTeX 以及 \href{http://www.latexstudio.net/}{LaTeX 工作室} 对于本系列模板的大力宣传与推广。

% 如果你喜欢我们的模板,你可以在 Github 上收藏我们的模板。

% \nocite{*}
% \printbibliography[heading=bibintoc, title=\ebibname]

% \appendix
% %\appendixpage
% \addappheadtotoc

\end{document}
