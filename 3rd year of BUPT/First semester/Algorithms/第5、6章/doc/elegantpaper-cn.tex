%!TEX program = xelatex
% 完整编译: xelatex -> biber/bibtex -> xelatex -> xelatex
\documentclass[lang=cn,11pt,a4paper]{elegantpaper}

\usepackage{algorithm}
\usepackage[noend]{algpseudocode}
\usepackage{amsmath}
\usepackage{array}
\usepackage{xcolor}
\usepackage{graphicx}
\usepackage{longtable}
\usepackage{bigstrut}


\lstdefinelanguage{text}{
    showstringspaces=false,
    breaklines=true,
    breakautoindent=true,
    breakindent=0em,
    escapeinside={(*@}{@*)},
    tabsize=4
}

\title{《算法设计与分析》\\课程实验报告}

\author{
\huge 专业:计算机科学与技术 \\[10pt]
\huge 班级:2021211304 \\[10pt]
\huge 姓名:杨晨 \\[10pt]
\huge 学号:2021212171
}
\date{}

% 本文档命令
\usepackage{array}
\newcommand{\ccr}[1]{\makecell{{\color{#1}\rule{1cm}{1cm}}}}

\begin{document}

\maketitle

\clearpage

% \begin{abstract}
% 本文为 \href{https://github.com/ElegantLaTeX/ElegantPaper/}{ElegantPaper} 的说明文档。此模板基于 \LaTeX{} 的 article 类,专为工作论文写作而设计。设计这个模板的初衷是让作者不用关心工作论文的格式,专心写作,从而有更加舒心的写作体验。如果你有其他问题、建议或者报告 bug,可以在 \href{https://github.com/ElegantLaTeX/ElegantPaper/issues}{GitHub::ElegantPaper/issues} 留言。如果你想了解更多 Elegant\LaTeX{} 项目组设计的模板,请访问 \href{https://github.com/ElegantLaTeX/}{GitHub::ElegantLaTeX}。
% \keywords{Elegant\LaTeX{},工作论文,模板}
% \end{abstract}


\section{概述}

\subsection{实验内容}

\begin{enumerate}
    \item 书面作业
    
    参照讲义PPT中(p26-28)给出的面向最大化问题(如0-1背包问题)的分支限界法算法框架,设计面向最小化问题,e.g. 旅行商问题,的分支限界法算法框架
    
    将算法框架附在实验报告中
    
    \item 编程作业
    
    采用回溯法、分支限界法,编程求解不同规模的旅行商问题TSP,并利用给定数据,验证算法正确性,对比算法的时间复杂性、空间复杂性

    \item 统计记录
    
    从起始城市出发的最短旅行路径
    
    路径总长度
    
    扫描过的搜索树结点总数L
    
    程序运行时间T
\end{enumerate}

\subsection{开发环境}

\begin{itemize}
    \item Windows10
    % \item PyCharm 2023.2.4 (Professional Edition)
    \item Visual Studio Code 1.84.2
\end{itemize}

\section{实验过程}

\subsection{回溯法求解TSP}

\subsubsection{介绍}
旅行商问题(Traveling Salesman Problem,TSP)是一个经典的组合优化问题,在计算机科学和运筹学领域有着广泛的应用。给定一组城市和它们之间的距离或成本,TSP的目标是找到一条最短路径,使得旅行商从起始城市出发,经过每个城市恰好一次,最后回到起始城市。

回溯法是一种解决组合优化问题的常用方法,它通过穷举所有可能的解空间,并利用剪枝策略来减少搜索空间的规模,从而找到最优解。在TSP问题中,回溯法通过递归地遍历所有可能的路径来找到最短路径。

\subsubsection{算法描述}
\begin{algorithm}[H]
\caption{TSP问题的回溯法求解}
\begin{algorithmic}[1]
\State sum\_node $\gets$ sum\_node + 1 \Comment{搜索节点数加1}

\If{count = |graph|}
    \If{graph[current\_city][start\_city] $\neq$ NO\_EDGE and now\_cost + graph[current\_city][start\_city] < min\_cost}
        \State min\_cost $\gets$ now\_cost + graph[current\_city][start\_city]
        \State best\_path $\gets$ path
        \State best\_path.append(start\_city)
    \EndIf
    \State \textbf{return}
\EndIf

\For{i 从 0 到 |graph|}
    \If{graph[current\_city][i] $\neq$ NO\_EDGE and !vis[i] and graph[current\_city][i] + now\_cost < min\_cost}
        \State vis[i] $\gets$ true \Comment{标记为已访问}
        \State path.append(i) \Comment{加入路径}
        \State TSP(count + 1, i, start\_city, min\_cost, now\_cost + graph[current\_city][i], best\_path, path, vis, graph, sum\_node)
        \State path.pop\_back() \Comment{回溯}
        \State vis[i] $\gets$ false
    \EndIf
\EndFor
\end{algorithmic}
\end{algorithm}

\subsubsection{分析和改进}
回溯法求解TSP问题的时间复杂度和空间复杂度如下:

\begin{itemize}
\item 时间复杂度:回溯法的时间复杂度是指数级别的,因为它需要遍历所有可能的解空间。在TSP问题中,假设有$n$个城市,则解空间的规模是$n!$。因此,回溯法的时间复杂度为$O(n!)$。然而,由于算法使用了剪枝策略(当当前花费加上当前节点到下一个节点的花费大于等于最小花费时,不再考虑某些状态),实际的时间复杂度可能会低于$O(n!)$。

\item 空间复杂度:回溯法的空间复杂度取决于递归调用的深度,即解空间的深度。在TSP问题中,递归调用的深度最多为$n$,即遍历所有城市一次。此外,还需要使用额外的空间存储路径、访问标记和邻接矩阵等信息。使用了两个长度为n的数组来记录当前的路径和访问状态,以及一个长度为n的数组来存储最优路径,其中n是城市的个数。,所以回溯法的空间复杂度为$O(n)$。请注意,这个分析是基于最坏情况的。由于剪枝策略的使用,实际的时间和空间需求可能会低于这个上限。
\end{itemize}

回溯法通过穷举搜索的方式可以找到TSP问题的最优解,但由于其时间复杂度的指数级增长,对于大规模问题可能会面临计算资源和时间的限制。在我的计算机上,对于超过20个节点的图,使用回溯法会爆栈,因此,我设计了模拟递归栈,消除递归的改进版本,详细代码见附录。


\subsection{分支限界法求解TSP}


\subsubsection{介绍}
旅行商问题(Traveling Salesman Problem,TSP)是一个经典的组合优化问题,目标是找到一条路径,使得旅行商从起始城市出发,经过所有城市恰好一次,然后返回起始城市,并且总路径长度最小。TSP在计算机科学和运筹学中具有重要的应用,它是一个NP困难问题,没有已知的多项式时间解法。分支限界法是一种常用的求解TSP问题的方法,通过不断剪枝和搜索空间的划分,寻找最优解。

\subsubsection{算法描述(含算法框架)}
算法框架以最小化问题为例,e.g. TSP问题
\begin{enumerate}
    \item 选择初始解对应的根节点$v_0$,根据限界函数,估计根节点的目标函数上下界, 确定目标函数的界[lowerBound, upperBound]问题最优的上下界
    \item 将活结点表ANT初始化为空
    \item 生成根结点$v_0$的全部子结点-宽度优先;

    对每个子结点$v$,执行以下操作
    
    \begin{enumerate}
        \item 估算$v$的目标函数值(下界)calculateLowerBound($v$)
        \item 若 calculateLowerBound($v$)<= upperBound,将$v$加入ANT表
    \end{enumerate}
    \item 循环,直到某个叶结点的目标函数值在表ANT中最小(找到1个具有最小值的完全解)
    \begin{enumerate}
        \item 从表ANT中选择(下界)lowerBound($v_i$)值最小的结点$v_i$ ,扩展其子结点(从活结点表中,选择1个具有最小可能目标值的扩展结点$v_i$)
        \item 对结点$v_i$的每个子结点$c$,执行下列操作
        \begin{enumerate}
            \item 估算c的目标函数值calculateLowerBound($v$),即下界
            \item 如果 calculateLowerBound($v$)<= upperBound,将c加入ANT表(子结点$c$有可能产生更优的解,将其加入活结点表,以后考虑对其进行扩展)
            \item 如果$c$是叶结点且lowerBound($c$)在表ANT中最小,则将结点$c$对应的完全解输出,算法结束(结点c对应了1个新找到的、具有最小目标值(e.g. TSP路径长度)的完全解——最优解)
            \item 如果$c$是叶结点但lowerBound($c$)在表ANT中不是最小,则:
            
            结点$c$对应了1个新找到的完全解,但该完全解的目标函数值与已经找到的、或未来可能找到完全解相比,并非更优
            
            i) upperBound = value($c$)
            
            ii) 对表ANT中所有满足calculateLowerBound($v_j$) > value($c$)的结点$v_j$,从表ANT中删除该结点!
        \end{enumerate}
    \end{enumerate}
\end{enumerate}
     


下面是使用分支限界法求解TSP问题的伪代码:

\begin{algorithm}[H]
\caption{TSP分支限界法}
\begin{algorithmic}[1]


\State Initialize priority queue $pq$ \Comment{优先队列,按照lowerBound从小到大排序}
\State Initialize boolean array $vis$ \Comment{标记节点是否被访问过}
\State Mark $start\_city$ as visited \Comment{起始节点标记为已访问}
\State Initialize empty path $path$ \Comment{当前路径}
\State Add $start\_city$ to $path$ \Comment{起始节点加入路径}
\State Calculate lowerBound \Comment{计算当前路径的下界}
\State Add State($start\_city$, 1, 0, lowerBound, $path$, $vis$) to $pq$ \Comment{起始节点加入优先队列}
\While{$pq$ is not empty}
    \State Get State $State$ from $pq$ \Comment{取出优先队列中的第一个状态}
    \State Remove $State$ from $pq$
    \State Increment $sum\_node$ \Comment{搜索节点数加1}
    \If{$State.lowerBound > upperBound$} \Comment{如果当前状态的下界大于上界,剪枝}
        \State Continue to next iteration
    \EndIf
    \If{$State.count = graph.size()$} \Comment{所有节点都访问过了,回到起始节点}
        \If{$graph[State.current\_city][start\_city] \neq NO\_EDGE$ and $State.now\_cost + graph[State.current\_city][start\_city] < min\_cost$}
            \State Update $min\_cost$, $best\_path$
            \State Add $start\_city$ to $best\_path$
            \If{$min\_cost \leq State.lowerBound$} \Comment{如果最小花费小于等于下界,直接返回,不再继续搜索,因为已经找到了最优解}
                \State Return
            \EndIf
            \If{$min\_cost < upperBound$} \Comment{如果最小花费小于上界,更新上界和pq}
               \State Update $upperBound$
               \State Update $pq$
            \EndIf
        \EndIf
        \State Continue to next iteration
    \EndIf
    \For{$i = 0$ to $graph.size()$}
        \If{$graph[State.current\_city][i] \neq NO\_EDGE$ and $!State.vis[i]$ and $graph[State.current\_city][i] + State.now\_cost < min\_cost$}
            \State Initialize new path $newPath$ \Comment{新路径}
            \State Copy $State.path$ to $newPath$ \Comment{加入下一个节点}
            \State Initialize new boolean array $newVis$ \Comment{新的标记数组}
            \State Copy $State.vis$ to $newVis$ \Comment{下一个节点标记为已访问}
            \State Calculate lowerBound
            \If{$lowerBound \leq upperBound$} \Comment{如果下界小于等于上界,加入优先队列}
                \State Add State($i$, $State.count + 1$, $State.now\_cost + graph[State.current\_city][i]$, $lowerBound$, $newPath$, $newVis$) to $pq$
            \EndIf
        \EndIf
    \EndFor
\EndWhile

\end{algorithmic}
\end{algorithm}

\subsubsection{算法分析}

\textbf{时间复杂度} 

算法的时间复杂度主要取决于两个部分:计算下界和搜索过程。

\begin{itemize}
    \item 计算下界:calculateLowerBound函数用于计算当前路径的下界。在最坏情况下,需要遍历每个节点,计算每个节点的最短路径和次短路径。由于预处理好了$min2$数组,因此时间复杂度为$O(n)$,其中n是节点的数量。
    \item 搜索过程:在最坏情况下,需要遍历所有可能的路径,即$n!$个排列(n是节点的数量)。每个节点都需要检查与其他节点的连边,因此时间复杂度为$O(n^2)$。因此,整个搜索过程的时间复杂度为$O(n! * n)$。
\end{itemize}

综上所述,算法的时间复杂度为$O(n! * n)$。然而,由于算法使用了剪枝策略(当下界大于上界时,不再考虑某些状态),实际的时间复杂度会低于$O(n! * n)$

\textbf{空间复杂度} 

算法使用了以下额外空间:

\begin{itemize}
    \item 优先队列:在最坏情况下,优先队列的大小可以达到n!,因此空间复杂度为$O(n!)$。
    \item vis数组、path数组和newPath数组:这些数组的大小与节点的数量n相同,因此空间复杂度为$O(n)$。
    \item min2二维数组和graph二维数组:这些二维数组的大小为$n * 2$ 和 $n * n$,因此空间复杂度为$O(n^2)$。
\end{itemize}

综上所述,算法的空间复杂度为O(n!),其中n是节点的数量。然而,这个分析是基于最坏情况的。由于剪枝策略的使用,实际的时间和空间需求可能会低于这个上限。此外,这个算法的效率也取决于 calculateLowerBound 函数的收益,该函数用于计算每个状态的下界。如果这个函数能够快速并准确地估计下界,那么算法的效率将会提高。

\section{实验结果}

\subsection{回溯法}

\begin{lstlisting}[language=text]
filename: 15.txt
min_cost: 5506.88
best_path: 20 9 7 16 3 13 12 21 10 8 19 11 22 5 17 20 
time: 0.0083363s
search node: 256955

filename: 20.txt
min_cost: 6987.51
best_path: 20 9 7 16 3 13 2 15 12 14 21 10 1 8 18 19 11 22 5 17 20 
time: 2.91684s
search node: 76329668

filename: 22.txt
min_cost: 7690.8
best_path: 20 9 7 16 3 13 2 15 12 14 21 10 1 4 6 18 8 19 11 22 5 17 20 
time: 20.1967s
search node: 487370492

filename: 30.txt
min_cost: 11426.6
best_path: 20 15 25 26 27 21 28 23 19 9 3 5 6 7 30 13 1 2 14 10 17 4 29 24 11 18 16 8 22 12 20 
time: 142.337s
search node: 3893952727

\end{lstlisting}

\subsection{分支限界法}
\begin{lstlisting}[language=text]
filename: 15.txt
min_cost: 5506.88
best_path: 20 17 5 22 11 19 8 10 21 12 13 3 16 7 9 20 
time: 0.0081651s
search node: 5422

filename: 20.txt
min_cost: 6987.51
best_path: 20 9 7 16 3 13 2 15 12 14 21 10 1 8 18 19 11 22 5 17 20 
time: 0.0539328s
search node: 28368

filename: 22.txt
min_cost: 7690.8
best_path: 20 9 7 16 3 13 2 15 12 14 21 10 1 4 6 18 8 19 11 22 5 17 20 
time: 0.0225743s
search node: 12594

filename: 30.txt
min_cost: 11426.6
best_path: 20 15 25 26 27 21 28 23 19 9 3 5 6 7 30 13 1 2 14 10 17 4 29 24 11 18 16 8 22 12 20 
time: 31.8229s
search node: 11135403

\end{lstlisting}

\subsection{表格记录}
\begin{longtable}{|c|c|m{2.5cm}|m{2cm}|m{2cm}|m{2cm}|}
    \hline
    问题 &  求解算法 & 最短回路 & 路径总长度(单位:m) & 搜索过的结点总数 & 程序运行时间(单位:s) \\
    \hline
    \multirow{2}{*}[-2.5ex]{\bigstrut 15个基站} & 回溯 & 20 9 7 16 3 13 12 21 10 8 19 11 22 5 17 20 & 5506.88 & 256955 & 0.0083363 \\
    \cline{2-6}
    & 分支限界 & 20 17 5 22 11 19 8 10 21 12 13 3 16 7 9 20 & 5506.88 & 5422 & 0.0081651 \\
    \hline
    \multirow{2}{*}[-4.5ex]{\bigstrut 20个基站} & 回溯 & 20 9 7 16 3 13 2 15 12 14 21 10 1 8 18 19 11 22 5 17 20 & 6987.51 & 76329668 & 2.91684 \\
    \cline{2-6}
    & 分支限界 & 20 9 7 16 3 13 2 15 12 14 21 10 1 8 18 19 11 22 5 17 20 & 6987.51 & 28368 & 0.0539328 \\
    \hline
    \multirow{2}{*}[-5.5ex]{\bigstrut 22个基站} & 回溯 & 20 9 7 16 3 13 2 15 12 14 21 10 1 4 6 18 8 19 11 22 5 17 20 & 7690.8 & 487370492 & 20.1967 \\
    \cline{2-6}
    & 分支限界 & 20 9 7 16 3 13 2 15 12 14 21 10 1 4 6 18 8 19 11 22 5 17 20 & 7690.8 & 12594 & 0.0225743 \\
    \hline
    \multirow{2}{*}[-8.5ex]{\bigstrut 30个基站} & 回溯 & 20 15 25 26 27 21 28 23 19 9 3 5 6 7 30 13 1 2 14 10 17 4 29 24 11 18 16 8 22 12 20 & 11426.6 & 3893952727 & 142.337 \\
    \cline{2-6}
    & 分支限界 & 20 15 25 26 27 21 28 23 19 9 3 5 6 7 30 13 1 2 14 10 17 4 29 24 11 18 16 8 22 12 20 & 11426.6 & 11135403 & 31.8229 \\
    \hline
\end{longtable}


\section{附录:完整代码}

\subsection{回溯法}
\begin{lstlisting}[language=c++]
#include <iostream>
#include <fstream>
#include <sstream>
#include <vector>
#include <unordered_map>
#include <stack>
#include <chrono>

#define NO_EDGE 99999

/*
 * fileName: 输入文件名
 * n: 节点数
 * graph: 邻接矩阵
 * id2index: id到下标的映射
 */
void input(const char *fileName, int &n, std::vector<std::vector<double>> &graph, std::unordered_map<int, int> &id2index)
{
    std::ifstream inputFile(fileName);
    if (!inputFile)
    {
        std::cout << "File not found!" << std::endl;
        return;
    }

    std::string line;
    std::getline(inputFile, line); // 读取第一行(编号行)
    for (int i = 0; i < line.size(); ++i)
        if (line[i] == '\t')
            ++n;
    n--;
    std::stringstream lineStream(line);

    for (int i = 0; i < n; ++i)
    {
        std::string temp;
        lineStream >> temp;
        id2index[i] = std::stoi(temp);
    }

    // 读取id行并忽略
    std::getline(inputFile, line);

    // 读取边的权值矩阵
    graph.resize(n, std::vector<double>(n));
    for (int i = 0; i < n; ++i)
    {
        std::getline(inputFile, line);
        std::stringstream lineStream(line);
        // 忽略前2列(编号和id)
        std::string temp;
        lineStream >> temp;
        lineStream >> temp;
        for (int j = 0; j < n; ++j)
        {
            lineStream >> graph[i][j];
        }
    }

    inputFile.close();
}

/*
 * count: 当前已经访问的节点数
 * current_city: 当前所在的节点
 * start_city: 起始节点
 * min_cost: 最小花费
 * now_cost: 当前花费
 * vis: 标记是否访问过
 * graph: 邻接矩阵
 * sum_node: 搜索节点数
 */
void TSP(int count, int current_city, int start_city, double &min_cost, double now_cost, std::vector<int> &best_path, std::vector<int> &path, std::vector<bool> &vis, std::vector<std::vector<double>> &graph, long long &sum_node)
{
    sum_node++;                // 搜索节点数加1
    if (count == graph.size()) // 所有节点都访问过了,回到起始节点
    {
        // 如果当前节点到起始节点有边,且当前花费加上当前节点到起始节点的花费小于最小花费
        if (graph[current_city][start_city] != NO_EDGE && now_cost + graph[current_city][start_city] < min_cost)
        {
            min_cost = now_cost + graph[current_city][start_city];
            best_path = path;
            best_path.push_back(start_city);
        }
        return;
    }

    for (int i = 0; i < graph.size(); ++i)
    {
        // 如果当前节点到下一个节点有边,且下一个节点未访问过,且当前花费加上当前节点到下一个节点的花费小于最小花费
        if (graph[current_city][i] != NO_EDGE && !vis[i] && graph[current_city][i] + now_cost < min_cost)
        {
            vis[i] = true;     // 标记为已访问
            path.push_back(i); // 加入路径
            TSP(count + 1, i, start_city, min_cost, now_cost + graph[current_city][i], best_path, path, vis, graph, sum_node);
            path.pop_back(); // 回溯
            vis[i] = false;
        }
    }
}

struct State // 栈中状态,引用变量不用拷贝
{
    int count;
    int current_city;
    int start_city;
    double now_cost;
    int last_index; // 上一个节点在path中的下标
    bool back_flag; // 是否是回溯
    // 构造函数
    State(int count, int current_city, int start_city, double now_cost, int last_index, bool back_flag) : count(count), current_city(current_city), start_city(start_city), now_cost(now_cost), last_index(last_index), back_flag(back_flag) {}
};
/*
 * count: 当前已经访问的节点数
 * current_city: 当前所在的节点
 * start_city: 起始节点
 * min_cost: 最小花费
 * now_cost: 当前花费
 * vis: 标记是否访问过
 * graph: 邻接矩阵
 */
long long TSP_stack(int count, int current_city, int start_city, double &min_cost, double now_cost, std::vector<int> &best_path, std::vector<int> &path, std::vector<bool> &vis, std::vector<std::vector<double>> &graph)
{
    long long sum_node = 0; // 搜索节点数
    std::stack<State, std::vector<State>> state_stack;
    state_stack.push(State(count, current_city, start_city, now_cost, 0, false));
    sum_node++;
    while (!state_stack.empty())
    {
        State &state = state_stack.top();
        switch (state.back_flag)
        {
        case false: // 前进
        {
            if (state.count == graph.size())
            {
                if (graph[state.current_city][state.start_city] != NO_EDGE && state.now_cost + graph[state.current_city][state.start_city] < min_cost)
                {
                    min_cost = state.now_cost + graph[state.current_city][state.start_city];
                    best_path = path;
                    best_path.push_back(state.start_city);
                }
                state_stack.top().back_flag = true;
                break;
            }
            bool forward_flag = false;
            for (int i = 0; i < graph.size(); ++i)
            {
                if (graph[state.current_city][i] != NO_EDGE && !vis[i] && graph[state.current_city][i] + state.now_cost < min_cost)
                {
                    vis[i] = true;     // 标记为已访问
                    path.push_back(i); // 加入路径
                    state_stack.push(State(state.count + 1, i, start_city, state.now_cost + graph[state.current_city][i], i, false));
                    forward_flag = true; // 有前进
                    sum_node++;          // 搜索节点数加1
                    break;
                }
            }
            if (!forward_flag) // 没有前进,回溯
                state_stack.top().back_flag = true;
            break;
        }
        case true: // 回溯
        {
            int last_index = state.last_index; // 上一个节点在path中的下标
            vis[last_index] = false;           // 标记为未访问
            path.pop_back();                   // 从路径中删除
            state_stack.pop();                 // 从栈中删除
            if (state_stack.empty())           // 栈空,退出
                break;
            state = state_stack.top(); // state维护的是栈顶元素的引用,所以要更新state
            bool forward_flag = false;
            for (int i = last_index + 1; i < graph.size(); ++i) // 从上一个节点的下一个节点开始找
            {
                if (graph[state.current_city][i] != NO_EDGE && !vis[i] && graph[state.current_city][i] + state.now_cost < min_cost)
                {
                    vis[i] = true;     // 标记为已访问
                    path.push_back(i); // 加入路径
                    state_stack.push(State(state.count + 1, i, start_city, state.now_cost + graph[state.current_city][i], i, false));
                    forward_flag = true; // 有前进
                    sum_node++;          // 搜索节点数加1
                    break;
                }
            }
            if (!forward_flag) // 没有前进,回溯
                state_stack.top().back_flag = true;
        }
        }
    }
    return sum_node; // 返回搜索节点数
}

// fileName: 输入文件名
void solve(const char *fileName, int start)
{
    std::cout << "filename: " << fileName << std::endl; // 输出文件名
    int n = 0;                                          // 节点数
    std::vector<std::vector<double>> graph;             // 邻接矩阵
    std::unordered_map<int, int> id2index;              // id到下标的映射
    input(fileName, n, graph, id2index);                // 读取输入文件
    double min_cost = NO_EDGE * n;                      // 最小花费
    std::vector<int> best_path(n + 1);                  // 最优路径,图总共n个点,回到起点又是一个点,所以是n+1个点
    int start_node = start;                             // 起始节点
    std::vector<int> path;                              // 当前路径
    std::vector<bool> vis(n, false);                    // 标记是否访问过
    vis[start_node] = true;                             // 起始节点标记为已访问
    path.push_back(start_node);                         // 起始节点加入路径

    long long sum_node = 0;                                      // 搜索节点数,初始化为0,longlong防止溢出
    auto start_time = std::chrono::high_resolution_clock::now(); // 计时开始
    // TSP(1, start_node, start_node, min_cost, 0, best_path, path, vis, graph, sum_node);
    sum_node = TSP_stack(1, start_node, start_node, min_cost, 0, best_path, path, vis, graph);
    auto end_time = std::chrono::high_resolution_clock::now();      // 计时结束
    std::chrono::duration<double> duration = end_time - start_time; // 计算耗时
    
    std::cout << "min_cost: " << min_cost << std::endl;
    std::cout << "best_path: ";
    for (int i = 0; i < best_path.size(); ++i) // 输出最短路径
    {
        std::cout << id2index[best_path[i]] << " "; // 输出id
        // std::cout << best_path[i] << " ";           // 输出下标
    }
    std::cout << std::endl
              << "time: " << duration.count() << "s" << std::endl;
    std::cout << "search node: " << sum_node << std::endl; // 输出搜索节点数
}

int main() // 由于用了大量stl,编译时请使用"-O1"或更高级优化选项
{
    freopen("output1.txt", "w", stdout);             // 将输出重定向到output.txt
    std::unordered_map<std::string, int> file2start; // 文件名到起始节点的映射
    file2start["15.txt"] = 12;
    file2start["20.txt"] = 17;
    file2start["22.txt"] = 19;
    file2start["30.txt"] = 19;

    solve("15.txt", file2start["15.txt"]);
    std::cout << std::endl;

    solve("20.txt", file2start["20.txt"]);
    std::cout << std::endl;

    solve("22.txt", file2start["22.txt"]);
    std::cout << std::endl;

    solve("30.txt", file2start["30.txt"]);
    std::cout << std::endl;
    fclose(stdout);
    return 0;
}
\end{lstlisting}

\subsection{分支限界法}

\begin{lstlisting}[language=c++]
#include <iostream>
#include <fstream>
#include <sstream>
#include <vector>
#include <unordered_map>
#include <algorithm>
#include <queue>
#include <chrono>

#define NO_EDGE 99999

/*
 * fileName: 输入文件名
 * n: 节点数
 * graph: 邻接矩阵
 * id2index: id到下标的映射
 */
void input(const char *fileName, int &n, std::vector<std::vector<double>> &graph, std::unordered_map<int, int> &id2index, std::vector<std::vector<double>> &min2)
{
    std::ifstream inputFile(fileName);
    if (!inputFile)
    {
        std::cout << "File not found!" << std::endl;
        return;
    }

    std::string line;
    std::getline(inputFile, line); // 读取第一行(编号行)
    for (int i = 0; i < line.size(); ++i)
        if (line[i] == '\t')
            ++n;
    n--;
    min2.resize(n, std::vector<double>(2, NO_EDGE)); // 初始化min2,n行2列,每个元素初始化为NO_EDGE
    std::stringstream lineStream(line);

    for (int i = 0; i < n; ++i)
    {
        std::string temp;
        lineStream >> temp;
        id2index[i] = std::stoi(temp);
    }

    // 读取id行并忽略
    std::getline(inputFile, line);

    // 读取边的权值矩阵
    graph.resize(n, std::vector<double>(n));
    for (int i = 0; i < n; ++i)
    {
        std::getline(inputFile, line);
        std::stringstream lineStream(line);
        // 忽略前2列(编号和id)
        std::string temp;
        lineStream >> temp;
        lineStream >> temp;
        for (int j = 0; j < n; ++j)
        {
            lineStream >> graph[i][j];
            if (graph[i][j] < min2[i][0]) // 比最小的小,更新最小的和次小的
            {
                min2[i][1] = min2[i][0];
                min2[i][0] = graph[i][j];
            }
            else if (graph[i][j] < min2[i][1]) // 比次小的小,更新次小的
            {
                min2[i][1] = graph[i][j];
            }
        }
    }

    inputFile.close();
}

/*
 * 求与当前节点最近的未访问过的节点的下标
 * graph: 邻接矩阵
 * vis: 标记节点是否被访问过
 * current_city: 当前所在的城市
 * exclude: 排除的节点
 * 返回值:下一个城市的下标
 */
int findMin(std::vector<std::vector<double>> &graph, std::vector<bool> &vis, int current_city, std::vector<bool> &exclude)
{
    double min_edge = NO_EDGE; // 最小边,初始化
    int next_city = -1;
    for (int i = 0; i < graph.size(); ++i)
    {
        // 如果当前节点到i有边小于最小值,且i没有被访问过,且i不在排除的节点中
        if (graph[current_city][i] < min_edge && !vis[i] && !exclude[i])
        {
            min_edge = graph[current_city][i]; // 维护最小边
            next_city = i;
        }
    }
    return next_city;
}

/*
 * 计算当前路径的上界
 * graph: 邻接矩阵
 * vis: 标记节点是否被访问过
 * current_city: 当前所在的城市
 * start_city: 起始城市
 * count: 已经访问过的节点数
 * 返回值:当前路径的上界
 */
double calculateUpperBound(std::vector<std::vector<double>> &graph, std::vector<bool> &vis, int current_city, int start_city, int count)
{
    if (count == graph.size())
    {
        if (graph[current_city][start_city] != NO_EDGE)
        {
            return graph[current_city][start_city];
        }
        else
            return -1;
    }
    std::vector<bool> exclude(graph.size(), false); // 排除的节点
    int next_city = findMin(graph, vis, current_city, exclude);
    while (next_city != -1)
    {
        vis[next_city] = true;
        double temp = calculateUpperBound(graph, vis, next_city, start_city, count + 1);
        if (temp != -1) // 找到了一条路径
            return temp + graph[current_city][next_city];
        vis[next_city] = false; // 回溯
        exclude[next_city] = true;
        next_city = findMin(graph, vis, current_city, exclude);
    }
    return -1; // 没有找到路径,返回-1,回溯
}

/*
 * 计算当前路径的下界
 * graph: 邻接矩阵
 * min2: 存储每个节点最短的2条路径的长度
 * path: 当前路径
 * 返回值:当前路径的下界
 */
double calculateLowerBound(std::vector<std::vector<double>> &graph, std::vector<std::vector<double>> &min2, std::vector<int> &path, std::vector<bool> &vis)
{
    int u = path[0], v = path[path.size() - 1]; // u,v记录当前路径的起点和终点
    double lowerBound = 0;
    int n = graph.size();
    for (int i = 0; i < path.size() - 1; i++) // 计算当前路径的花费
        lowerBound += graph[path[i]][path[i + 1]];
    lowerBound *= 2;
    if (path.size() >= 2) // 路径中有2个点以上
    {
        double min_edge1 = NO_EDGE, min_edge2 = NO_EDGE; // 分别记录回到u的最小边和从v出发的最小边
        for (int i = 0; i < n; i++)
        {
            if (graph[i][u] < min_edge1) 
                min_edge1 = graph[i][u];
            if (graph[v][i] < min_edge2)
                min_edge2 = graph[v][i];
        }
        lowerBound += min_edge1 + min_edge2;
    }
    else // 只有1个点,直接加上最小的2条边
    {
        lowerBound += min2[u][0] + min2[u][1];
    }
    for (int i = 0; i < n; i++)
    {
        if (!vis[i]) // 如果i不在当前路径中
        {
            lowerBound += min2[i][0] + min2[i][1]; // 加上i的最小的2条边
        }
    }
    return lowerBound / 2;
}

struct State
{
    int current_city;      // 当前所在的城市
    int count;             // 已经访问过的节点数
    double now_cost;       // 当前路径的花费
    double lowerBound;     // 当前路径的下界
    std::vector<int> path; // 当前路径
    std::vector<bool> vis; // 标记节点是否被访问过
    // 构造函数
    State(int current_city, int count, double now_cost, double lowerBound, std::vector<int> &path, std::vector<bool> &vis)
        : current_city(current_city), count(count), now_cost(now_cost), lowerBound(lowerBound), path(path), vis(vis) {}

    // 重载大于号,按照lowerBound从小到大排序
    friend bool operator>(const State &s1, const State &s2)
    {
        return s1.lowerBound > s2.lowerBound;
    }
};

/*
 * start_city: 起始节点
 * min_cost: 最小花费
 * best_path: 最优路径
 * graph: 邻接矩阵
 * min2: 存储每个节点最短的2条路径的长度
 * sum_node: 搜索节点数
 * upperBound: 上界
 */
void TSP(int start_city, double &min_cost, std::vector<int> &best_path, std::vector<std::vector<double>> &graph, std::vector<std::vector<double>> &min2, long long &sum_node, double upperBound)
{
    std::priority_queue<State, std::vector<State>, std::greater<State>> pq; // 优先队列,按照lowerBound从小到大排序
    std::vector<bool> vis(graph.size(), false);                             // 标记节点是否被访问过
    vis[start_city] = true;                                                 // 起始节点标记为已访问
    std::vector<int> path;                                                  // 当前路径
    path.push_back(start_city);                                             // 起始节点加入路径
    double lowerBound = calculateLowerBound(graph, min2, path, vis);        // 计算当前路径的下界
    // std::cout << "lowerBound: " << lowerBound << std::endl;                 // 输出下界
    pq.push(State(start_city, 1, 0, lowerBound, path, vis));                // 起始节点加入优先队列
    while (!pq.empty())
    {
        State state = pq.top(); // 取出优先队列中的第一个状态
        pq.pop();
        sum_node++;                        // 搜索节点数加1
        if (state.lowerBound > upperBound) // 如果当前状态的下界大于上界,剪枝
            continue;
        if (state.count == graph.size()) // 所有节点都访问过了,回到起始节点
        {
            // 如果当前节点到起始节点有边,且当前花费加上当前节点到起始节点的花费小于最小花费
            if (graph[state.current_city][start_city] != NO_EDGE && state.now_cost + graph[state.current_city][start_city] < min_cost)
            {
                min_cost = state.now_cost + graph[state.current_city][start_city];
                best_path = state.path;
                best_path.push_back(start_city);
                if (min_cost <= state.lowerBound) // 如果最小花费小于等于下界,直接返回,不再继续搜索,因为已经找到了最优解
                    return;
                if (min_cost < upperBound) // 如果最小花费小于上界,更新上界,并删除优先队列中大于当前最小花费的状态
                {
                    upperBound = min_cost;
                    std::vector<State> temp;
                    while (!pq.empty())
                    {
                        State s = pq.top();
                        pq.pop();
                        if (s.lowerBound < upperBound)
                            temp.push_back(s);
                    }
                    for (int i = 0; i < temp.size(); ++i)
                        pq.push(temp[i]);
                }
                    
            }
            continue;
        }
        for (int i = 0; i < graph.size(); ++i)
        {
            // 如果当前节点到下一个节点有边,且下一个节点未访问过,且当前花费加上当前节点到下一个节点的花费小于最小花费
            if (graph[state.current_city][i] != NO_EDGE && !state.vis[i] && graph[state.current_city][i] + state.now_cost < min_cost)
            {
                std::vector<int> newPath = state.path; // 新路径
                newPath.push_back(i);                  // 加入下一个节点
                std::vector<bool> newVis = state.vis;  // 新的标记数组
                newVis[i] = true;                      // 下一个节点标记为已访问
                double lowerBound = calculateLowerBound(graph, min2, newPath, newVis);
                if (lowerBound <= upperBound) // 如果下界小于等于上界,加入优先队列
                    pq.push(State(i, state.count + 1, state.now_cost + graph[state.current_city][i], lowerBound, newPath, newVis));
            }
        }
    }
}

void solve(const char *fileName, int start)
{
    std::cout << "filename: " << fileName << std::endl;                             // 输出文件名
    int n = 0, start_node = start;                                                  // 节点数,起始节点
    std::vector<std::vector<double>> graph;                                         // 邻接矩阵
    std::unordered_map<int, int> id2index;                                          // id到下标的映射
    std::vector<std::vector<double>> min2;                                          // 存储每个节点最短的2条路径的长度
    input(fileName, n, graph, id2index, min2);                                      // 读取输入文件
    std::vector<bool> vis(n, false);                                                // 标记节点是否被访问过
    vis[start_node] = true;                                                         // 起始节点标记为已访问
    double upperBound = calculateUpperBound(graph, vis, start_node, start_node, 1); // 计算上界
    // std::cout << "upperBound: " << upperBound << std::endl;                         // 输出上界

    long long sum_node = 0;                                      // 搜索节点数,初始化为0,longlong防止溢出
    double min_cost = NO_EDGE * n;                               // 最小花费
    std::vector<int> best_path(n + 1);                           // 最优路径,图总共n个点,回到起点又是一个点,所以是n+1个点
    vis[start_node] = true;                                      // 起始节点标记为已访问
    auto start_time = std::chrono::high_resolution_clock::now(); // 计时开始
    TSP(start_node, min_cost, best_path, graph, min2, sum_node, upperBound);
    auto end_time = std::chrono::high_resolution_clock::now();      // 计时结束
    std::chrono::duration<double> duration = end_time - start_time; // 计算耗时

    std::cout << "min_cost: " << min_cost << std::endl;
    std::cout << "best_path: ";
    for (int i = 0; i < best_path.size(); ++i)
        std::cout << id2index[best_path[i]] << " ";
    std::cout << std::endl
              << "time: " << duration.count() << "s" << std::endl;
    std::cout << "search node: " << sum_node << std::endl; // 输出搜索节点数
}

int main() // 使用了大量stl,编译时请使用"-O1"或更高级优化选项
{
    freopen("output2.txt", "w", stdout);             // 输出重定向到output.txt
    std::unordered_map<std::string, int> file2start; // 文件名到起始节点的映射
    file2start["15.txt"] = 12;
    file2start["20.txt"] = 17;
    file2start["22.txt"] = 19;
    file2start["30.txt"] = 19;

    solve("15.txt", file2start["15.txt"]);
    std::cout << std::endl;

    solve("20.txt", file2start["20.txt"]);
    std::cout << std::endl;

    solve("22.txt", file2start["22.txt"]);
    std::cout << std::endl;

    solve("30.txt", file2start["30.txt"]);
    std::cout << std::endl;
    fclose(stdout);
    return 0;
}
\end{lstlisting}

\end{document}
